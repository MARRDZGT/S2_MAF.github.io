\documentclass{exam}
\usepackage[T1]{fontenc}
\usepackage[activeacute,spanish]{babel}
\usepackage[utf8]{inputenc}
\usepackage{graphicx}
\usepackage{textcomp}
\usepackage{amssymb}
\usepackage{amsmath, amsfonts, amsthm}


\begin{document}

\begin{center}
\fbox{\fbox{\parbox{6.5in}{\centering
{\large Matemáticas Avanzadas de la Física}\\
\textbf{\large Semanal 4}}}}
\end{center}

\vspace{3mm}
\makebox[0.9\textwidth]{Semana de Entrega: 4}

%\date{\displaydate{date}}

\vspace{5mm}

\begin{questions}

\question Una sucesión $(x_n)$ en un espacio métrico $(X,d)$ ---------- o es ------------- a $x \in X$ si

\[\lim_{ n \to \infty} d(x_n, x) = 0\]

donde al punto $x$ se le llama el límite de la sucesión $(x_n)$

\begin{itemize}
\item[a)] Converge o es divergente
\item[b)] Converge o es convergente
\item[c)] Diverge o es convergente
\item[d)] Diverge o es divergente
\end{itemize}

Respuesta buena: ¡Muy bien! esa es la definición de convergencia\\
Respuesta erronea: ¡error!  cuidado con los juegos de palabras.

\question La sucesión $1/n$ (si, hasta el cansancio) es un buen ejemplo de una sucesión divergente, sin embargo existe el límite: su límite es $0$. ¿Qué condición necesita para ser convergente? 



\begin{itemize}
\item[a)] La sucesión no converge en $X$, por lo tanto su límite está en $X$
\item[b)] Para que la sucesión sea convergente en $X$, se necesita que el límite esté en $X$
\item[c)] La sucesión simplemente diverge
\item[d)] La sucesión es convergente
\end{itemize}

Respuesta buena: ¡Muy bien! si el límite está en el espacio, la sucesión es convergente\\
Respuesta erronea: Cuidado, podemos hacer que esta sucesión sea convergente, necesitamos que su límite esté en el espacio métrico.

\question Sea $X$ un espacio vectorial sobre el campo. Una norma es una función $||.|| : X \to [0,\infty)$ si y sólo si

\[||x|| \geq 0 \quad \text{y} \quad ||x|| = 0 \Leftrightarrow x = 0\]

a esta propiedad se le conoce

\begin{itemize}
\item[a)] La norma de un subespacio
\item[b)] La norma es definida positiva
\item[c)] Homogeneidad del valor absoluto
\item[d)] Desigualdad del triángulo
\end{itemize}

Respuesta buena: ¡Muy bien! la norma por definición debe ser positiva\\
Respuesta erronea: ¡error! recuerda las tres propiedades de un espacio normado.

\question Sea $X$ un espacio vectorial sobre el campo. Una norma es una función $||.|| : X \to [0,\infty)$ si y sólo si

\[||\lambda x||  = |\lambda| ||x|| \quad \forall \lambda \in \mathbb{K} \quad \text{y} \quad x \in X\]

a esta propiedad se le conoce

\begin{itemize}
\item[a)] La norma de un subespacio
\item[b)] La norma es definida positiva
\item[c)] Homogeneidad del valor absoluto
\item[d)] Desigualdad del triángulo
\end{itemize}

Respuesta buena: ¡Muy bien! esta es la homogeneidad del valor absoluto\\
Respuesta erronea: ¡error! recuerda las tres propiedades de un espacio normado.

\question Sea $X$ un espacio vectorial sobre el campo. Una norma es una función $||.|| : X \to [0,\infty)$ si y sólo si

\[||x+y|| \leq ||x|| + ||y|| \quad \forall x, y \in X  \]

a esta propiedad se le conoce

\begin{itemize}
\item[a)] La norma de un subespacio
\item[b)] La norma es definida positiva
\item[c)] Homogeneidad del valor absoluto
\item[d)] Desigualdad del triángulo
\end{itemize}

Respuesta buena: ¡Muy bien! esta es la desigualdad del triángulo para normas\\
Respuesta erronea: ¡error! recuerda las tres propiedades de un espacio normado.

\question Sea $(X, ||\cdot||)$ un espacio normado. La norma da la noción de ``------'' a cada vector del espacio vectorial e induce una métrica $d$ sobre $X$, dada por

\[d_{||\cdot||}(x,y) = ||x-y|| \]

\begin{itemize}
\item[a)] Longitud
\item[b)] Distancia
\item[c)] Espacio
\item[d)] Continuidad
\end{itemize}

Respuesta buena: ¡Muy bien! la norma da la noción de longitud, así como la métrica da la noción de distancia\\
Respuesta erronea: ¡error! si la métrica da noción de distancia ¿qué nos dice la norma?

\question Si el espacio normado $(X, ||\cdot||)$ es un espacio de Banach ¿cuáles son las condiciones para que esto suceda?

\begin{itemize}
\item[a)] Que el espacio tenga estructura de espacio vectorial y que sea completo (Toda sucesión de elementos del espacio es de Cauchy)
\item[b)] Que el espacio sea métrico y topológico
\item[c)] Que el espacio sea vectorial y defina una métrica
\item[d)] Que toda sucesión de Cauchy sea convergente en el espacio.
\end{itemize}

Respuesta buena: ¡Muy bien! Es de Banach.\\
Respuesta erronea: ¡error! Piensa en la completez de espacios normados.

\question la suma y el producto de normas, pueden escribirse como una generalización de la desigualdad del trińgulo, para el caso del producto

\[\sum_{j=1}^{n} |x_j y_j| \leq  \left(\sqrt{\sum_{k=1}^{n}|x_k|^p}\right)^{\frac{1}{p}} \left(\sqrt{\sum_{m=1}^{n}|y_m|^q}\right) ^{\frac{1}{q}}\]

con $p > 1$ y $\frac{1}{p} + \frac{1}{q} = 1$

y la suma

\[ \left(\sum_{j=1}^{n} |x_j + y_ j|^p \left)^{\frac{1}{p}} \leq \left(\sqrt{\sum_{k=1}^{n}|x_k|^p}\right)^{\frac{1}{p}} + \left(\sqrt{\sum_{m=1}^{n}|y_m|^p}\right)^{\frac{1}{p}}\]

donde $p \geq 1$. Son conocidas como:
 
  
\begin{itemize}
\item[a)] Desigualdad del triángulo generalizada
\item[b)] Desigualdades $l^p$
\item[c)] Desigualdad de Hölder y Minkowsky
\item[d)] Desigualdad de Cauchy y Minkowsky
\end{itemize}

Respuesta buena: ¡Muy bien! Son las desigualdades de Hölder y Minkowsky\\
Respuesta erronea: ¡error! La prueba está en la tarea.

\question Completa: Del teorema: ``Cada subespacio de dimensión ------------ y de un espacio normado $X$ es completo. En partícular, cada espacio normado de dimensión -------- es completo.'' 

\begin{itemize}
\item[a)] Dimensión finita
\item[b)] Dimensión infinita
\item[c)] Dimensión infinita y finita
\item[d)] Dimensión finita e infinita.
\end{itemize}

Respuesta buena: ¡Muy bien! Un espacio normado de dimensión finita es completo\
Respuesta erronea: ¡error! lee con cuidado.


\question Sea una sucesión infinita de términos $a_1, a_2, ... $. Se define su suma parcial como

\[S_i = \sum _{n=1}^{i} a_n\]

Si la suma parcial converge a cierto límite finito cuando $i \to \infty$, la serie infinita se dice 

\begin{itemize}
\item[a)] Convergente y debe cumplir el critero de Cauchy
\item[b)] Convergente y esta condición es suficiente
\item[c)] Divergente y necesita cumplir el critero de Cauchy
\item[d)] No dice nada
\end{itemize}

Respuesta buena: ¡Muy bien! La serie tiende a cierto límite pero debe de cumplir el criterio de convergencia de Cauchy (condición necesaria y suficiente)\\
Respuesta erronea: ¡error! la suma infinita debe ser de Cauchy y tener límite en el espacio.
\end{questions}
\end{document}