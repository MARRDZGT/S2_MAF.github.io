\documentclass{exam}
\usepackage[T1]{fontenc}
\usepackage[activeacute,spanish]{babel}
\usepackage[utf8]{inputenc}
\usepackage{graphicx}
\usepackage{textcomp}
\usepackage{amssymb}
\usepackage{amsmath, amsfonts, amsthm}


\begin{document}

\begin{center}
\fbox{\fbox{\parbox{6.5in}{\centering
{\large Matemáticas Avanzadas de la Física}\\
\textbf{\large Semanal 2}}}}
\end{center}

\vspace{3mm}
\makebox[0.9\textwidth]{Semana de Entrega: 2}

%\date{\displaydate{date}}

\vspace{5mm}

\begin{questions}

\question Sea $x_0 \in X$ y $r>0$ un real. Definimos:

\[B(x_0; r) = \lbrace x \in X | d(x,x_0) < r \]

como:

\begin{itemize}
\item[a)] Como la Bola abierta con centro en $x_0$ y radio $r$
\item[b)] Como la Bola cerrada con centro en $x_0$ y radio $r$
\item[c)] Como Esfera con centro en $x_0$ y radio $r$
\item[d)] El conjunto vacío $\emptyset$
\end{itemize}

Respuesta buena: ¡Muy bien! esa es la definición de Bola abierta\\
Respuesta erronea: ¡error!  observa bien las desigualdades

\question Sea $x_0 \in X$ y $r>0$ un real. Definimos:

\[\overline{B}(x_0; r) = \lbrace x \in X | d(x,x_0) \leq r \]

como:

\begin{itemize}
\item[a)] Como la Bola abierta con centro en $x_0$ y radio $r$
\item[b)] Como la Bola cerrada con centro en $x_0$ y radio $r$
\item[c)] Como Esfera con centro en $x_0$ y radio $r$
\item[d)] El conjunto vacío $\emptyset$
\end{itemize}

Respuesta buena: ¡Muy bien! esa es la definición de Bola cerrada\\
Respuesta erronea: ¡error! observa bien las desigualdades

\question Sea $x_0 \in X$ y $r>0$ un real. Definimos:

\[S(x_0; r) = \lbrace x \in X | d(x,x_0) = r \rbrace\]

como:

\begin{itemize}
\item[a)] Como la Bola abierta con centro en $x_0$ y radio $r$
\item[b)] Como la Bola cerrada con centro en $x_0$ y radio $r$
\item[c)] Como Esfera con centro en $x_0$ y radio $r$
\item[d)] El conjunto vacío $\emptyset$
\end{itemize}

Respuesta buena: ¡Muy bien! esa es la definición de esfera\\
Respuesta erronea: ¡error! observa bien la igualdad

\question Sea $A \subset X$ con $x \in X$ se dice que para todo $\epsilon >0 $

\[B(x; \epsilon) \cap A \neq \emptyset\]

\begin{itemize}
\item[a)] Es un punto frontera o de acumulación de A
\item[b)] Es un punto frontera o la frontera de A
\item[c)] Es la cerradura de A
\item[d)] Es un punto interior de A
\end{itemize}

Respuesta buena: ¡Muy bien! es la definición de punto de acumulación\\
Respuesta erronea: ¡Error!

\question Sea $A \subset X$ con $x \in X$ se dice que para todo $\epsilon >0 $

\[\partial A : = \lbrace x \in X | x \quad \text{es un punto frontera de} \quad A \rbrace\]

\begin{itemize}
\item[a)] Es un punto frontera o de acumulación de A
\item[b)] Es un punto frontera o la frontera de A
\item[c)] Es la cerradura de A
\item[d)] Es un punto interior de A
\end{itemize}

Respuesta buena: ¡Muy bien! es la definición de punto de acumulación\\
Respuesta erronea: ¡Error!

\question Completa la frase: ``Las bolas abiertas son ------------------------- y las bolas cerradas son -------------------''

\begin{itemize}
\item[a)] Intervalos
\item[b)] Conjuntos abiertos y conjuntos cerrados
\item[c)] Bolas semi abiertas y semi cerradas
\item[d)] Bases de vecindades
\end{itemize}

Respuesta buena: ¡Muy bien! los conjuntos heredan propiedades algebraicas en bolas\\
Respuesta erronea: ¡error! lee con cuidado.


\question Completa la frase: ``Todos los espacios métricos son --------------, la -------- define la noción de ---------, pero no todos los espacios métricos son topológicos. En general, un espacio topológico $(X, \tau)$ es metrizable si existe una métrica $d$ en $X$ que induce la topologı́a $\tau$.''

\begin{itemize}
\item[a)] normados, norma, vector
\item[b)] topológicos, métrica, bola
\item[c)] métricos, topología, intervalo
\item[d)] con producto interno, norma, vectores
\end{itemize}

Respuesta buena: ¡Muy bien! los espacios métricos son espacios topológicos, aunque no ocurre en sentido contrario\\
Respuesta erronea: ¡error! lee con cuidado


\question El siguiente teorema: ``Para todo $n \in \mathbb{N}$ y $A \subset \mathbb{R}^{n}$ con la topoloǵia euclidiana. $A$ es compacto si y sólo si, $A$ es cerrado y acotado'' corresponde a

\begin{itemize}
\item[a)] Teorema espacio métrico segundo numerable
\item[b)] Teorema de Heine-Borel
\item[c)] Teorema continuidad topológica
\item[d)] Teorema continuidad en espacios métricos.
\end{itemize}

Respuesta buena: ¡Muy bien! el teorema de Heine-Borel habla de subconjuntos compactos sobre campos reales o complejos\\
Respuesta erronea: ¡Error! este teorema fue fundamental en el curso de cálculo vectorial (Cálculo 3)

\end{questions}

\end{document}